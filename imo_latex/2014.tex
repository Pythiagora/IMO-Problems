\documentclass[12pt]{article}
\usepackage{amsmath, amssymb}
\title{IMO 2014}
\date{}
\begin{document}
\maketitle

\textbf{Problem 1.}
Let $a_0 < a_1 < a_2 < \cdots$ be an infinite sequence of positive integers. Prove that there exists a unique integer $n \geq 1$ such that
\[
a_n < \frac{a_0 + a_1 + \cdots + a_n}{n} \leq a_{n+1}.
\]

\bigskip

\textbf{Problem 2.}
Let $n \geq 2$ be an integer. Consider an $n \times n$ chessboard consisting of $n^2$ unit squares. A configuration of $n$ rooks on this board is \emph{peaceful} if every row and every column contains exactly one rook. Find the greatest positive integer $k$ such that, for each peaceful configuration of $n$ rooks, there is a $k \times k$ square which does not contain a rook on any of its $k^2$ unit squares.

\bigskip

\textbf{Problem 3.}
Convex quadrilateral $ABCD$ has $\angle ABC = \angle CDA = 90^\circ$. Point $H$ is the foot of the perpendicular from $A$ to $BD$. Points $S$ and $T$ lie on sides $AB$ and $AD$, respectively, such that $H$ lies inside triangle $SCT$ and
\[
\angle CHS - \angle CSB = 90^\circ, \qquad \angle THC - \angle DTC = 90^\circ.
\]
Prove that line $BD$ is tangent to the circumcircle of triangle $TSH$.

\bigskip

\textbf{Problem 4.}
Points $P$ and $Q$ lie on side $BC$ of acute-angled triangle $ABC$ so that $\angle PAB = \angle BCA$ and $\angle CAQ = \angle ABC$. Points $M$ and $N$ lie on lines $AP$ and $AQ$, respectively, such that $P$ is the midpoint of $AM$, and $Q$ is the midpoint of $AN$. Prove that lines $BM$ and $CN$ intersect on the circumcircle of triangle $ABC$.

\bigskip

\textbf{Problem 5.}
For each positive integer $n$, the Bank of Cape Town issues coins of denomination $\frac{1}{n}$. Given a finite collection of such coins (of not necessarily different denominations) with total value at most $99 + \frac{1}{2}$, prove that it is possible to split this collection into $100$ or fewer groups, such that each group has total value at most $1$.

\bigskip

\textbf{Problem 6.}
A set of lines in the plane is in \emph{general position} if no two are parallel and no three pass through the same point. A set of lines in general position cuts the plane into regions, some of which have finite area; we call these its \emph{finite regions}. Prove that for all sufficiently large $n$, in any set of $n$ lines in general position it is possible to colour at least $\sqrt{n}$ of the lines blue in such a way that none of its finite regions has a completely blue boundary.

\end{document}

\documentclass[12pt]{article}
\usepackage{amsmath, amssymb}
\title{IMO 2022}
\date{}
\begin{document}
\maketitle

\textbf{Problem 1.}
The Bank of Oslo issues two types of coin: aluminium (denoted A) and bronze (denoted B). Marianne has $n$ aluminium coins and $n$ bronze coins, arranged in a row in some arbitrary initial order. A \textit{chain} is any subsequence of consecutive coins of the same type. Given a fixed positive integer $k \leq 2n$, Marianne repeatedly performs the following operation: she identifies the longest chain containing the $k$th coin from the left, and moves all coins in that chain to the left end of the row. For example, if $n = 4$ and $k = 4$, the process starting from the ordering $AABBBABA$ would be
\[
AABBBABA \to BBBAAABA \to AAABBBBA \to BBBBAAAA \to BBBBAAAA \to \cdots.
\]
Find all pairs $(n, k)$ with $1 \leq k \leq 2n$ such that for every initial ordering, at some moment during the process, the leftmost $n$ coins will all be of the same type.

\bigskip

\textbf{Problem 2.}
Let $\mathbb{R}^+$ denote the set of positive real numbers. Find all functions $f \colon \mathbb{R}^+ \to \mathbb{R}^+$ such that for each $x \in \mathbb{R}^+$, there is exactly one $y \in \mathbb{R}^+$ satisfying
\[
xf(y) + yf(x) \leq 2.
\]

\bigskip

\textbf{Problem 3.}
Let $k$ be a positive integer and let $S$ be a finite set of odd prime numbers. Prove that there is at most one way (up to rotation and reflection) to place the elements of $S$ around a circle such that the product of any two neighbours is of the form $x^2 + x + k$ for some positive integer $x$.

\bigskip

\textbf{Problem 4.}
Let $ABCDE$ be a convex pentagon such that $BC = DE$. Assume that there is a point $T$ inside $ABCDE$ with $TB = TD$, $TC = TE$ and $\angle ABT = \angle TEA$. Let line $AB$ intersect lines $CD$ and $CT$ at points $P$ and $Q$, respectively. Assume that the points $P$, $B$, $A$, $Q$ occur on their line in that order. Let line $AE$ intersect lines $CD$ and $DT$ at points $R$ and $S$, respectively. Assume that the points $R$, $E$, $A$, $S$ occur on their line in that order. Prove that the points $P$, $S$, $Q$, $R$ lie on a circle.

\bigskip

\textbf{Problem 5.}
Find all triples $(a, b, p)$ of positive integers with $p$ prime and
\[
a^p = b! + p.
\]

\bigskip

\textbf{Problem 6.}
Let $n$ be a positive integer. A \textit{Nordic square} is an $n \times n$ board containing all the integers from $1$ to $n^2$ so that each cell contains exactly one number. Two different cells are considered \textit{adjacent} if they share a common side. Every cell that is adjacent only to cells containing larger numbers is called a \textit{valley}. An \textit{uphill path} is a sequence of one or more cells such that:
\begin{enumerate}
  \item[(i)] the first cell in the sequence is a valley,
  \item[(ii)] each subsequent cell in the sequence is adjacent to the previous cell, and
  \item[(iii)] the numbers written in the cells in the sequence are in increasing order.
\end{enumerate}
Find, as a function of $n$, the smallest possible total number of uphill paths in a Nordic square.

\end{document}

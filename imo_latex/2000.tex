\documentclass[12pt]{article}
\usepackage{amsmath, amssymb}
\title{IMO 2000}
\date{}
\begin{document}
\maketitle

\textbf{Problem 1.}
$AB$ is tangent to the circles $CAMN$ and $NMBD$. $M$ lies between $C$ and $D$ on the line $CD$, and $CD$ is parallel to $AB$. The chords $NA$ and $CM$ meet at $P$; the chords $NB$ and $MD$ meet at $Q$. The rays $CA$ and $DB$ meet at $E$. Prove that $PE = QE$.

\bigskip

\textbf{Problem 2.}
$A$, $B$, $C$ are positive reals with product $1$. Prove that
\[
\left(A - 1 + \frac{1}{B}\right)\left(B - 1 + \frac{1}{C}\right)\left(C - 1 + \frac{1}{A}\right) \leq 1.
\]

\bigskip

\textbf{Problem 3.}
$k$ is a positive real. $N$ is an integer greater than $1$. $N$ points are placed on a line, not all coincident. A move is carried out as follows: pick any two points $A$ and $B$ which are not coincident. Suppose that $A$ lies to the right of $B$. Replace $B$ by another point $B'$ to the right of $A$ such that $AB' = k \cdot BA$. For what values of $k$ can we move the points arbitrarily far to the right by repeated moves?

\bigskip

\textbf{Problem 4.}
$100$ cards are numbered $1$ to $100$ (each card different) and placed in $3$ boxes (at least one card in each box). How many ways can this be done so that if two boxes are selected and a card is taken from each, then the knowledge of their sum alone is always sufficient to identify the third box?

\bigskip

\textbf{Problem 5.}
Can we find $N$ divisible by just $2000$ different primes, so that $N$ divides $2^N + 1$? ($N$ may be divisible by a prime power.)

\bigskip

\textbf{Problem 6.}
$A_1 A_2 A_3$ is an acute-angled triangle. The foot of the altitude from $A_i$ is $K_i$ and the incircle touches the side opposite $A_i$ at $L_i$. The line $K_1 K_2$ is reflected in the line $L_1 L_2$. Similarly, the line $K_2 K_3$ is reflected in $L_2 L_3$ and $K_3 K_1$ is reflected in $L_3 L_1$. Show that the three new lines form a triangle with vertices on the incircle.

\end{document}

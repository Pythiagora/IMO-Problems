\documentclass[12pt]{article}
\usepackage{amsmath, amssymb}
\title{IMO 2008}
\date{}
\begin{document}
\maketitle

\textbf{Problem 1.}
An acute-angled triangle $ABC$ has orthocentre $H$. The circle passing through $H$ with centre the midpoint of $BC$ intersects the line $BC$ at $A_1$ and $A_2$. Similarly, the circle passing through $H$ with centre the midpoint of $CA$ intersects the line $CA$ at $B_1$ and $B_2$, and the circle passing through $H$ with centre the midpoint of $AB$ intersects the line $AB$ at $C_1$ and $C_2$. Show that $A_1$, $A_2$, $B_1$, $B_2$, $C_1$, $C_2$ lie on a circle.

\bigskip

\textbf{Problem 2.}
\begin{enumerate}
\item[(a)] Prove that
\[
\frac{x^2}{(x-1)^2} + \frac{y^2}{(y-1)^2} + \frac{z^2}{(z-1)^2} \geq 1
\]
for all real numbers $x, y, z$, each different from $1$, and satisfying $xyz = 1$.
\item[(b)] Prove that equality holds above for infinitely many triples of rational numbers $x, y, z$, each different from $1$, and satisfying $xyz = 1$.
\end{enumerate}

\bigskip

\textbf{Problem 3.}
Prove that there exist infinitely many positive integers $n$ such that $n^2 + 1$ has a prime divisor which is greater than $2n + \sqrt{2n}$.

\bigskip

\textbf{Problem 4.}
Find all functions $f \colon (0, \infty) \to (0, \infty)$ such that
\[
\frac{\bigl(f(w)\bigr)^2 + \bigl(f(x)\bigr)^2}{f(y^2) + f(z^2)} = \frac{w^2 + x^2}{y^2 + z^2}
\]
for all positive real numbers $w, x, y, z$ satisfying $wx = yz$.

\bigskip

\textbf{Problem 5.}
Let $n$ and $k$ be positive integers with $k \geq n$ and $k - n$ an even number. Let $2n$ lamps labelled $1, 2, \ldots, 2n$ be given, each of which can be either on or off. Initially all the lamps are off. We consider sequences of steps: at each step one of the lamps is switched (from on to off or from off to on).

Let $N$ be the number of such sequences consisting of $k$ steps and resulting in the state where lamps $1$ through $n$ are all on, and lamps $n+1$ through $2n$ are all off.

Let $M$ be the number of such sequences consisting of $k$ steps, resulting in the state where lamps $1$ through $n$ are all on, and lamps $n+1$ through $2n$ are all off, but where none of the lamps $n+1$ through $2n$ is ever switched on.

Determine the ratio $N/M$.

\bigskip

\textbf{Problem 6.}
Let $ABCD$ be a convex quadrilateral with $|BA| \neq |BC|$. Denote the incircles of triangles $ABC$ and $ADC$ by $\omega_1$ and $\omega_2$ respectively. Suppose that there exists a circle $\omega$ tangent to the ray $BA$ beyond $A$ and to the ray $BC$ beyond $C$, which is also tangent to the lines $AD$ and $CD$. Prove that the common external tangents of $\omega_1$ and $\omega_2$ intersect on $\omega$.

\end{document}

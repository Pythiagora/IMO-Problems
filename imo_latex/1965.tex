\documentclass[12pt]{article}
\usepackage{amsmath, amssymb}
\title{IMO 1965}
\date{}
\begin{document}
\maketitle

\textbf{Problem 1.}
Determine all values $x$ in the interval $0 \leq x \leq 2\pi$ which satisfy the inequality
\[
\sqrt{2}\cos x \leq \left|\sqrt{1 + \sin 2x} - \sqrt{1 - \sin 2x}\right| \leq \sqrt{2}.
\]

\bigskip

\textbf{Problem 2.}
Consider the system of equations
\[
a_{11}x_1 + a_{12}x_2 + a_{13}x_3 = 0,\quad
a_{21}x_1 + a_{22}x_2 + a_{23}x_3 = 0,\quad
a_{31}x_1 + a_{32}x_2 + a_{33}x_3 = 0,
\]
with unknowns $x_1, x_2, x_3$. The coefficients satisfy the conditions:
\begin{enumerate}
\item[(a)] $a_{11}, a_{22}, a_{33}$ are positive numbers;
\item[(b)] the remaining coefficients are negative numbers;
\item[(c)] in each equation, the sum of the coefficients is positive.
\end{enumerate}
Prove that the given system has only the solution $x_1 = x_2 = x_3 = 0$.

\bigskip

\textbf{Problem 3.}
Given the tetrahedron $ABCD$ whose edges $AB$ and $CD$ have lengths $a$ and $b$ respectively. The distance between the skew lines $AB$ and $CD$ is $d$, and the angle between them is $\omega$. Tetrahedron $ABCD$ is divided into two solids by plane $\varepsilon$, parallel to lines $AB$ and $CD$. The ratio of the distances of $\varepsilon$ from $AB$ and $CD$ is equal to $k$. Compute the ratio of the volumes of the two solids obtained.

\bigskip

\textbf{Problem 4.}
Find all sets of four real numbers $x_1, x_2, x_3, x_4$ such that the sum of any one and the product of the other three is equal to 2.

\bigskip

\textbf{Problem 5.}
Consider $\triangle OAB$ with acute angle $AOB$. Through a point $M \neq O$ perpendiculars are drawn to $OA$ and $OB$, the feet of which are $P$ and $Q$ respectively. The point of intersection of the altitudes of $\triangle OPQ$ is $H$. What is the locus of $H$ if $M$ is permitted to range over (a) the side $AB$, (b) the interior of $\triangle OAB$?

\bigskip

\textbf{Problem 6.}
In a plane a set of $n$ points ($n \geq 3$) is given. Each pair of points is connected by a segment. Let $d$ be the length of the longest of these segments. We define a diameter of the set to be any connecting segment of length $d$. Prove that the number of diameters of the given set is at most $n$.

\end{document}

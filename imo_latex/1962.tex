\documentclass[12pt]{article}
\usepackage{amsmath, amssymb}
\title{IMO 1962}
\date{}
\begin{document}
\maketitle

\textbf{Problem 1.}
Find the smallest natural number $n$ which has the following properties:
\begin{enumerate}
\item[(a)] Its decimal representation has 6 as the last digit.
\item[(b)] If the last digit 6 is erased and placed in front of the remaining digits, the resulting number is four times as large as the original number $n$.
\end{enumerate}

\bigskip

\textbf{Problem 2.}
Determine all real numbers $x$ which satisfy the inequality:
\[
\sqrt{3-x} - \sqrt{x+1} > \frac{1}{2}.
\]

\bigskip

\textbf{Problem 3.}
Consider the cube $ABCDA'B'C'D'$ ($ABCD$ and $A'B'C'D'$ are the upper and lower bases, respectively, and edges $AA'$, $BB'$, $CC'$, $DD'$ are parallel). The point $X$ moves at constant speed along the perimeter of the square $ABCD$ in the direction $A \to B \to C \to D \to A$, and the point $Y$ moves at the same rate along the perimeter of the square $B'C'CB$ in the direction $B' \to C' \to C \to B \to B'$. Points $X$ and $Y$ begin their motion at the same instant from the starting positions $A$ and $B'$, respectively. Determine and draw the locus of the midpoints of the segments $XY$.

\bigskip

\textbf{Problem 4.}
Solve the equation $\cos^2 x + \cos^2 2x + \cos^2 3x = 1$.

\bigskip

\textbf{Problem 5.}
On the circle $K$ there are given three distinct points $A$, $B$, $C$. Construct (using only straightedge and compasses) a fourth point $D$ on $K$ such that a circle can be inscribed in the quadrilateral thus obtained.

\bigskip

\textbf{Problem 6.}
Consider an isosceles triangle. Let $r$ be the radius of its circumscribed circle and $\rho$ the radius of its inscribed circle. Prove that the distance $d$ between the centers of these two circles is
\[
d = \sqrt{r(r - 2\rho)}.
\]

\bigskip

\textbf{Problem 7.}
The tetrahedron $SABC$ has the following property: there exist five spheres, each tangent to the edges $SA$, $SB$, $SC$, $BC$, $CA$, $AB$, or to their extensions.
\begin{enumerate}
\item[(a)] Prove that the tetrahedron $SABC$ is isosceles (i.e., that all four faces are congruent triangles).
\item[(b)] Prove conversely that for every isosceles tetrahedron five such spheres exist.
\end{enumerate}

\end{document}

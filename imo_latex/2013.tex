\documentclass[12pt]{article}
\usepackage{amsmath, amssymb}
\title{IMO 2013}
\date{}
\begin{document}
\maketitle

\textbf{Problem 1.}
Prove that for any pair of positive integers $k$ and $n$, there exist $k$ positive integers $m_1, m_2, \ldots, m_k$ (not necessarily different) such that
\[
1 + \frac{2^k - 1}{n} = \left(1 + \frac{1}{m_1}\right)\left(1 + \frac{1}{m_2}\right) \cdots \left(1 + \frac{1}{m_k}\right).
\]

\bigskip

\textbf{Problem 2.}
A configuration of $4027$ points in the plane is called \emph{Colombian} if it consists of $2013$ red points and $2014$ blue points, and no three of the points of the configuration are collinear. By drawing some lines, the plane is divided into several regions. An arrangement of lines is \emph{good} for a Colombian configuration if the following two conditions are satisfied:
\begin{itemize}
\item no line passes through any point of the configuration;
\item no region contains points of both colours.
\end{itemize}
Find the least value of $k$ such that for any Colombian configuration of $4027$ points, there is a good arrangement of $k$ lines.

\bigskip

\textbf{Problem 3.}
Let the excircle of triangle $ABC$ opposite the vertex $A$ be tangent to the side $BC$ at the point $A_1$. Define the points $B_1$ on $CA$ and $C_1$ on $AB$ analogously, using the excircles opposite $B$ and $C$, respectively. Suppose that the circumcentre of triangle $A_1B_1C_1$ lies on the circumcircle of triangle $ABC$. Prove that triangle $ABC$ is right-angled.

The excircle of triangle $ABC$ opposite the vertex $A$ is the circle that is tangent to the line segment $BC$, to the ray $AB$ beyond $B$, and to the ray $AC$ beyond $C$. The excircles opposite $B$ and $C$ are similarly defined.

\bigskip

\textbf{Problem 4.}
Let $ABC$ be an acute-angled triangle with orthocentre $H$, and let $W$ be a point on the side $BC$, lying strictly between $B$ and $C$. The points $M$ and $N$ are the feet of the altitudes from $B$ and $C$, respectively. Denote by $\omega_1$ the circumcircle of $BWN$, and let $X$ be the point on $\omega_1$ such that $WX$ is a diameter of $\omega_1$. Analogously, denote by $\omega_2$ the circumcircle of $CWM$, and let $Y$ be the point on $\omega_2$ such that $WY$ is a diameter of $\omega_2$. Prove that $X$, $Y$ and $H$ are collinear.

\bigskip

\textbf{Problem 5.}
Let $\mathbb{Q}_{>0}$ be the set of positive rational numbers. Let $f \colon \mathbb{Q}_{>0} \to \mathbb{R}$ be a function satisfying the following three conditions:
\begin{enumerate}
\item[(i)] for all $x, y \in \mathbb{Q}_{>0}$, we have $f(x)f(y) \geq f(xy)$;
\item[(ii)] for all $x, y \in \mathbb{Q}_{>0}$, we have $f(x + y) \geq f(x) + f(y)$;
\item[(iii)] there exists a rational number $a > 1$ such that $f(a) = a$.
\end{enumerate}
Prove that $f(x) = x$ for all $x \in \mathbb{Q}_{>0}$.

\bigskip

\textbf{Problem 6.}
Let $n \geq 3$ be an integer, and consider a circle with $n + 1$ equally spaced points marked on it. Consider all labellings of these points with the numbers $0, 1, \ldots, n$ such that each label is used exactly once; two such labellings are considered to be the same if one can be obtained from the other by a rotation of the circle. A labelling is called \emph{beautiful} if, for any four labels $a < b < c < d$ with $a + d = b + c$, the chord joining the points labelled $a$ and $d$ does not intersect the chord joining the points labelled $b$ and $c$.

Let $M$ be the number of beautiful labellings, and let $N$ be the number of ordered pairs $(x, y)$ of positive integers such that $x + y \leq n$ and $\gcd(x, y) = 1$. Prove that
\[
M = N + 1.
\]

\end{document}

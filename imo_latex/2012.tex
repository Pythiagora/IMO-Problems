\documentclass[12pt]{article}
\usepackage{amsmath, amssymb}
\title{IMO 2012}
\date{}
\begin{document}
\maketitle

\textbf{Problem 1.}
Given triangle $ABC$ the point $J$ is the centre of the excircle opposite the vertex $A$. This excircle is tangent to the side $BC$ at $M$, and to the lines $AB$ and $AC$ at $K$ and $L$, respectively. The lines $LM$ and $BJ$ meet at $F$, and the lines $KM$ and $CJ$ meet at $G$. Let $S$ be the point of intersection of the lines $AF$ and $BC$, and let $T$ be the point of intersection of the lines $AG$ and $BC$. Prove that $M$ is the midpoint of $ST$.

(The excircle of $ABC$ opposite the vertex $A$ is the circle that is tangent to the line segment $BC$, to the ray $AB$ beyond $B$, and to the ray $AC$ beyond $C$.)

\bigskip

\textbf{Problem 2.}
Let $n \geq 3$ be an integer, and let $a_2, a_3, \ldots, a_n$ be positive real numbers such that $a_2 a_3 \cdots a_n = 1$. Prove that
\[
(1 + a_2)^2 (1 + a_3)^3 \cdots (1 + a_n)^n > n^n.
\]

\bigskip

\textbf{Problem 3.}
The liar's guessing game is a game played between two players $A$ and $B$. The rules of the game depend on two positive integers $k$ and $n$ which are known to both players.

At the start of the game $A$ chooses integers $x$ and $N$ with $1 \leq x \leq N$. Player $A$ keeps $x$ secret, and truthfully tells $N$ to player $B$. Player $B$ now tries to obtain information about $x$ by asking player $A$ questions as follows: each question consists of $B$ specifying an arbitrary set $S$ of positive integers (possibly one specified in some previous question), and asking $A$ whether $x$ belongs to $S$. Player $B$ may ask as many such questions as he wishes. After each question, player $A$ must immediately answer it with yes or no, but is allowed to lie as many times as she wants; the only restriction is that, among any $k+1$ consecutive answers, at least one answer must be truthful.

After $B$ has asked as many questions as he wants, he must specify a set $X$ of at most $n$ positive integers. If $x$ belongs to $X$, then $B$ wins; otherwise, he loses. Prove that:
\begin{enumerate}
\item[1.] If $n \geq 2^k$, then $B$ can guarantee a win.
\item[2.] For all sufficiently large $k$, there exists an integer $n \geq 1.99^k$ such that $B$ cannot guarantee a win.
\end{enumerate}

\bigskip

\textbf{Problem 4.}
Find all functions $f \colon \mathbb{Z} \to \mathbb{Z}$ such that, for all integers $a, b, c$ that satisfy $a + b + c = 0$, the following equality holds:
\[
f(a)^2 + f(b)^2 + f(c)^2 = 2f(a)f(b) + 2f(b)f(c) + 2f(c)f(a).
\]
(Here $\mathbb{Z}$ denotes the set of integers.)

\bigskip

\textbf{Problem 5.}
Let $ABC$ be a triangle with $\angle BCA = 90^\circ$, and let $D$ be the foot of the altitude from $C$. Let $X$ be a point in the interior of the segment $CD$. Let $K$ be the point on the segment $AX$ such that $BK = BC$. Similarly, let $L$ be the point on the segment $BX$ such that $AL = AC$. Let $M$ be the point of intersection of $AL$ and $BK$.

Show that $MK = ML$.

\bigskip

\textbf{Problem 6.}
Find all positive integers $n$ for which there exist non-negative integers $a_1, a_2, \ldots, a_n$ such that
\[
\frac{1}{2^{a_1}} + \frac{1}{2^{a_2}} + \cdots + \frac{1}{2^{a_n}} = \frac{a_1}{3} + \frac{a_2}{3^2} + \cdots + \frac{a_n}{3^n} = 1.
\]

\end{document}

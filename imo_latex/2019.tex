\documentclass[12pt]{article}
\usepackage{amsmath, amssymb}
\title{IMO 2019}
\date{}
\begin{document}
\maketitle

\textbf{Problem 1.}
Let $\mathbb{Z}$ be the set of integers. Determine all functions $f \colon \mathbb{Z} \to \mathbb{Z}$ such that, for all integers $a$ and $b$,
\[
f(2a) + 2f(b) = f(f(a + b)).
\]

\bigskip

\textbf{Problem 2.}
In triangle $ABC$, point $A_1$ lies on side $BC$ and point $B_1$ lies on side $AC$. Let $P$ and $Q$ be points on segments $AA_1$ and $BB_1$, respectively, such that $PQ$ is parallel to $AB$. Let $P_1$ be a point on line $PB_1$, such that $B_1$ lies strictly between $P$ and $P_1$, and $\angle PP_1C = \angle BAC$. Similarly, let $Q_1$ be a point on line $QA_1$, such that $A_1$ lies strictly between $Q$ and $Q_1$, and $\angle CQ_1Q = \angle CBA$.

Prove that points $P$, $Q$, $P_1$, and $Q_1$ are concyclic.

\bigskip

\textbf{Problem 3.}
A social network has $2019$ users, some pairs of whom are friends. Whenever user $A$ is friends with user $B$, user $B$ is also friends with user $A$. Events of the following kind may happen repeatedly, one at a time:

\begin{quote}
Three users $A$, $B$, and $C$ such that $A$ is friends with both $B$ and $C$, but $B$ and $C$ are not friends, change their friendship statuses such that $B$ and $C$ are now friends, but $A$ is no longer friends with $B$, and no longer friends with $C$. All other friendship statuses are unchanged.
\end{quote}

Initially, $1010$ users have $1009$ friends each, and $1009$ users have $1010$ friends each. Prove that there exists a sequence of such events after which each user is friends with at most one other user.

\bigskip

\textbf{Problem 4.}
Find all pairs $(k, n)$ of positive integers such that
\[
k! = (2^n - 1)(2^n - 2)(2^n - 4) \cdots (2^n - 2^{n-1}).
\]

\bigskip

\textbf{Problem 5.}
The Bank of Bath issues coins with an H on one side and a T on the other. Harry has $n$ of these coins arranged in a line from left to right. He repeatedly performs the following operation: if there are exactly $k > 0$ coins showing H, then he turns over the $k$th coin from the left; otherwise, all coins show T and he stops. For example, if $n = 3$ the process starting with the configuration $THT$ would be $THT \to HHT \to HTT \to TTT$, which stops after three operations.

\begin{enumerate}
  \item[(a)] Show that, for each initial configuration, Harry stops after a finite number of operations.
  \item[(b)] For each initial configuration $C$, let $L(C)$ be the number of operations before Harry stops. For example, $L(THT) = 3$ and $L(TTT) = 0$. Determine the average value of $L(C)$ over all $2^n$ possible initial configurations $C$.
\end{enumerate}

\bigskip

\textbf{Problem 6.}
Let $I$ be the incentre of acute triangle $ABC$ with $AB \neq AC$. The incircle $\omega$ of $ABC$ is tangent to sides $BC$, $CA$, and $AB$ at $D$, $E$, and $F$, respectively. The line through $D$ perpendicular to $EF$ meets $\omega$ again at $R$. Line $AR$ meets $\omega$ again at $P$. The circumcircles of triangles $PCE$ and $PBF$ meet again at $Q$.

Prove that lines $DI$ and $PQ$ meet on the line through $A$ perpendicular to $AI$.

\end{document}

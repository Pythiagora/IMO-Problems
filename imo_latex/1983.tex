\documentclass[12pt]{article}
\usepackage{amsmath, amssymb}
\title{IMO 1983}
\date{}
\begin{document}
\maketitle

\textbf{Problem 1.}
Find all functions $f$ defined on the set of positive real numbers which take positive real values and satisfy the conditions:
\begin{enumerate}
    \item[(i)] $f(xf(y)) = yf(x)$ for all positive $x, y$;
    \item[(ii)] $f(x) \to 0$ as $x \to \infty$.
\end{enumerate}

\bigskip

\textbf{Problem 2.}
Let $A$ be one of the two distinct points of intersection of two unequal coplanar circles $C_1$ and $C_2$ with centers $O_1$ and $O_2$, respectively. One of the common tangents to the circles touches $C_1$ at $P_1$ and $C_2$ at $P_2$, while the other touches $C_1$ at $Q_1$ and $C_2$ at $Q_2$. Let $M_1$ be the midpoint of $P_1 Q_1$, and $M_2$ be the midpoint of $P_2 Q_2$. Prove that $\angle O_1 A O_2 = \angle M_1 A M_2$.

\bigskip

\textbf{Problem 3.}
Let $a$, $b$ and $c$ be positive integers, no two of which have a common divisor greater than $1$. Show that $2abc - ab - bc - ca$ is the largest integer which cannot be expressed in the form $xbc + yca + zab$, where $x$, $y$ and $z$ are non-negative integers.

\bigskip

\textbf{Problem 4.}
Let $ABC$ be an equilateral triangle and $E$ the set of all points contained in the three segments $AB$, $BC$ and $CA$ (including $A$, $B$ and $C$). Determine whether, for every partition of $E$ into two disjoint subsets, at least one of the two subsets contains the vertices of a right-angled triangle. Justify your answer.

\bigskip

\textbf{Problem 5.}
Is it possible to choose $1983$ distinct positive integers, all less than or equal to $10^5$, no three of which are consecutive terms of an arithmetic progression? Justify your answer.

\bigskip

\textbf{Problem 6.}
Let $a$, $b$ and $c$ be the lengths of the sides of a triangle. Prove that
\[
a^2 b(a - b) + b^2 c(b - c) + c^2 a(c - a) \geq 0.
\]
Determine when equality occurs.

\end{document}

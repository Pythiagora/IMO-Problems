\documentclass[12pt]{article}
\usepackage{amsmath, amssymb}
\title{IMO 2009}
\date{}
\begin{document}
\maketitle

\textbf{Problem 1.}
Let $n$ be a positive integer and let $a_1, \ldots, a_k$ ($k \geq 2$) be distinct integers in the set $\{1, \ldots, n\}$ such that $n$ divides $a_i(a_{i+1} - 1)$ for $i = 1, \ldots, k-1$. Prove that $n$ does not divide $a_k(a_1 - 1)$.

\bigskip

\textbf{Problem 2.}
Let $ABC$ be a triangle with circumcentre $O$. The points $P$ and $Q$ are interior points of the sides $CA$ and $AB$, respectively. Let $K$, $L$ and $M$ be the midpoints of the segments $BP$, $CQ$ and $PQ$, respectively, and let $\Gamma$ be the circle passing through $K$, $L$ and $M$. Suppose that the line $PQ$ is tangent to the circle $\Gamma$. Prove that $OP = OQ$.

\bigskip

\textbf{Problem 3.}
Suppose that $s_1, s_2, s_3, \ldots$ is a strictly increasing sequence of positive integers such that the subsequences
\[
s_{s_1},\, s_{s_2},\, s_{s_3},\, \ldots \quad \text{and} \quad s_{s_1+1},\, s_{s_2+1},\, s_{s_3+1},\, \ldots
\]
are both arithmetic progressions. Prove that the sequence $s_1, s_2, s_3, \ldots$ is itself an arithmetic progression.

\bigskip

\textbf{Problem 4.}
Let $ABC$ be a triangle with $AB = AC$. The angle bisectors of $\angle CAB$ and $\angle ABC$ meet the sides $BC$ and $CA$ at $D$ and $E$, respectively. Let $K$ be the incentre of triangle $ADC$. Suppose that $\angle BEK = 45^\circ$. Find all possible values of $\angle CAB$.

\bigskip

\textbf{Problem 5.}
Determine all functions $f$ from the set of positive integers to the set of positive integers such that, for all positive integers $a$ and $b$, there exists a non-degenerate triangle with sides of lengths $a$, $f(b)$ and $f(b + f(a) - 1)$.

(A triangle is non-degenerate if its vertices are not collinear.)

\bigskip

\textbf{Problem 6.}
Let $a_1, a_2, \ldots, a_n$ be distinct positive integers and let $M$ be a set of $n-1$ positive integers not containing $s = a_1 + a_2 + \cdots + a_n$. A grasshopper is to jump along the real axis, starting at the point $0$ and making $n$ jumps to the right with lengths $a_1, a_2, \ldots, a_n$ in some order. Prove that the order can be chosen in such a way that the grasshopper never lands on any point in $M$.

\end{document}

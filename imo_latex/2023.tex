\documentclass[12pt]{article}
\usepackage{amsmath, amssymb}
\title{IMO 2023}
\date{}
\begin{document}
\maketitle

\textbf{Problem 1.}
Determine all composite integers $n > 1$ that satisfy the following property: if $d_1, d_2, \ldots, d_k$ are all the positive divisors of $n$ with $1 = d_1 < d_2 < \cdots < d_k = n$, then $d_i$ divides $d_{i+1} + d_{i+2}$ for every $1 \leq i \leq k - 2$.

\bigskip

\textbf{Problem 2.}
Let $ABC$ be an acute-angled triangle with $AB < AC$. Let $\Omega$ be the circumcircle of $ABC$. Let $S$ be the midpoint of the arc $CB$ of $\Omega$ containing $A$. The perpendicular from $A$ to $BC$ meets $BS$ at $D$ and meets $\Omega$ again at $E \neq A$. The line through $D$ parallel to $BC$ meets line $BE$ at $L$. Denote the circumcircle of triangle $BDL$ by $\omega$. Let $\omega$ meet $\Omega$ again at $P \neq B$.

Prove that the line tangent to $\omega$ at $P$ meets line $BS$ on the internal angle bisector of $\angle BAC$.

\bigskip

\textbf{Problem 3.}
For each integer $k \geq 2$, determine all infinite sequences of positive integers $a_1, a_2, \ldots$ for which there exists a polynomial $P$ of the form
\[
P(x) = x^k + c_{k-1} x^{k-1} + \cdots + c_1 x + c_0,
\]
where $c_0, c_1, \ldots, c_{k-1}$ are non-negative integers, such that
\[
P(a_n) = a_{n+1} a_{n+2} \cdots a_{n+k}
\]
for every integer $n \geq 1$.

\bigskip

\textbf{Problem 4.}
Let $x_1, x_2, \ldots, x_{2023}$ be pairwise different positive real numbers such that
\[
a_n = (x_1 + x_2 + \cdots + x_n)\left(\frac{1}{x_1} + \frac{1}{x_2} + \cdots + \frac{1}{x_n}\right)
\]
is an integer for every $n = 1, 2, \ldots, 2023$. Prove that $a_{2023} \geq 3034$.

\bigskip

\textbf{Problem 5.}
Let $n$ be a positive integer. A \textit{Japanese triangle} consists of $1 + 2 + \cdots + n$ circles arranged in an equilateral triangular shape such that for each $i = 1, 2, \ldots, n$, the $i$th row contains exactly $i$ circles, exactly one of which is coloured red. A \textit{ninja path} in a Japanese triangle is a sequence of $n$ circles obtained by starting in the top row, then repeatedly going from a circle to one of the two circles immediately below it and finishing in the bottom row.

In terms of $n$, find the greatest $k$ such that in each Japanese triangle there is a ninja path containing at least $k$ red circles.

\bigskip

\textbf{Problem 6.}
Let $ABC$ be an equilateral triangle. Let $A_1$, $B_1$, $C_1$ be interior points of $ABC$ such that $BA_1 = A_1C$, $CB_1 = B_1A$, $AC_1 = C_1B$, and
\[
\angle BA_1C + \angle CB_1A + \angle AC_1B = 480^\circ.
\]
Let $BC_1$ and $CB_1$ meet at $A_2$, let $CA_1$ and $AC_1$ meet at $B_2$, and let $AB_1$ and $BA_1$ meet at $C_2$. Prove that if triangle $A_1B_1C_1$ is scalene, then the three circumcircles of triangles $AA_1A_2$, $BB_1B_2$ and $CC_1C_2$ all pass through two common points.

(\textit{Note:} a scalene triangle is one where no two sides have equal length.)

\end{document}

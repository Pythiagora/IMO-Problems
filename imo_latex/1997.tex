\documentclass[12pt]{article}
\usepackage{amsmath, amssymb}
\title{IMO 1997}
\date{}
\begin{document}
\maketitle

\textbf{Problem 1.}
In the plane the points with integer coordinates are the vertices of unit squares. The squares are colored alternately black and white (as on a chessboard). For any pair of positive integers $m$ and $n$, consider a right-angled triangle whose vertices have integer coordinates and whose legs, of lengths $m$ and $n$, lie along edges of the squares.

Let $S_1$ be the total area of the black part of the triangle and $S_2$ be the total area of the white part. Let
\[
f(m, n) = |S_1 - S_2|.
\]
\begin{enumerate}
\item[(a)] Calculate $f(m, n)$ for all positive integers $m$ and $n$ which are either both even or both odd.
\item[(b)] Prove that $f(m, n) \leq \frac{1}{2}\max\{m, n\}$ for all $m$ and $n$.
\item[(c)] Show that there is no constant $C$ such that $f(m, n) < C$ for all $m$ and $n$.
\end{enumerate}

\bigskip

\textbf{Problem 2.}
The angle at $A$ is the smallest angle of triangle $ABC$. The points $B$ and $C$ divide the circumcircle of the triangle into two arcs. Let $U$ be an interior point of the arc between $B$ and $C$ which does not contain $A$. The perpendicular bisectors of $AB$ and $AC$ meet the line $AU$ at $V$ and $W$, respectively. The lines $BV$ and $CW$ meet at $T$. Show that
\[
AU = TB + TC.
\]

\bigskip

\textbf{Problem 3.}
Let $x_1, x_2, \ldots, x_n$ be real numbers satisfying the conditions
\[
|x_1 + x_2 + \cdots + x_n| = 1
\]
and
\[
|x_i| \leq \frac{n+1}{2}, \qquad i = 1, 2, \ldots, n.
\]
Show that there exists a permutation $y_1, y_2, \ldots, y_n$ of $x_1, x_2, \ldots, x_n$ such that
\[
|y_1 + 2y_2 + \cdots + ny_n| \leq \frac{n+1}{2}.
\]

\bigskip

\textbf{Problem 4.}
An $n \times n$ matrix whose entries come from the set $S = \{1, 2, \ldots, 2n-1\}$ is called a \emph{silver matrix} if, for each $i = 1, 2, \ldots, n$, the $i$th row and the $i$th column together contain all elements of $S$. Show that
\begin{enumerate}
\item[(a)] there is no silver matrix for $n = 1997$;
\item[(b)] silver matrices exist for infinitely many values of $n$.
\end{enumerate}

\bigskip

\textbf{Problem 5.}
Find all pairs $(a, b)$ of integers $a, b \geq 1$ that satisfy the equation
\[
a^{b^2} = b^a.
\]

\bigskip

\textbf{Problem 6.}
For each positive integer $n$, let $f(n)$ denote the number of ways of representing $n$ as a sum of powers of $2$ with nonnegative integer exponents. Representations which differ only in the ordering of their summands are considered to be the same. For instance, $f(4) = 4$, because the number $4$ can be represented in the following four ways:
\[
4; \quad 2+2; \quad 2+1+1; \quad 1+1+1+1.
\]
Prove that, for any integer $n \geq 3$,
\[
2^{n^2/4} < f(2^n) < 2^{n^2/2}.
\]

\end{document}
